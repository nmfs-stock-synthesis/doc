% Preamble =========================================================================================
%Options added to improve pdf accessibility:
\RequirePackage{pdfmanagement-testphase}
\PassOptionsToPackage{enable-debug,check-declarations}{expl3}
%\DeclareDocumentMetadata {  }
\DocumentMetadata{testphase={phase-III,math}}
\ExplSyntaxOn
\pdfmanagement_add:nnn{Catalog}{Lang}{(enUS)}
\ExplSyntaxOff
% End accessibility options
\documentclass[12pt]{article}
\usepackage{sectsty}               % allows for different fonts for header and body
\usepackage{natbib}                % bibliography package
\usepackage[margin=1in, includefoot]{geometry}
\usepackage{graphicx}              % allows for image import
\usepackage{enumerate}
% \usepackage{multirow}              % multi-row in tables
% \usepackage{multicol}              % multi-column in tables
\usepackage{booktabs}
\usepackage{lmodern}
\usepackage[none]{hyphenat}
% \usepackage{array}
% \usepackage{lscape}
% \usepackage{pdflscape}
% \usepackage{longtable}
\usepackage[utf8]{inputenc}
\usepackage[T1]{fontenc}         	% controls font encoding
\usepackage{roboto} 				% sans serif font for headers
\usepackage{crimson}              	% serif font
\usepackage[table]{xcolor}
\usepackage{colortbl}
% \usepackage{hhline}
% \usepackage{dcolumn}
\usepackage{tocloft}
\setlength{\cftsubsecnumwidth}{3em}
\setlength{\cftsubsubsecnumwidth}{4em}
\usepackage{amsmath}
\usepackage{unicode-math}
\usepackage{float}
\usepackage{fancyhdr}
\usepackage[parfill]{parskip}
\usepackage[nottoc,numbib]{tocbibind}
\usepackage[all]{nowidow}		    % Widow Control
% \usepackage{hyperref}
% \usepackage{hypcap}

% For tagging the pdf: https://github.com/jgm/pandoc/issues/5409#issuecomment-770417614
\ifluatex
%   \usepackage[luamode]{tagpdf}
  \tagpdfsetup{
      activate-all=true,
      interwordspace=true
   }
\fi
% End tagging the pdf

% Set widow and club penalties
\widowpenalty=100000
\clubpenalty=100000

%sectsty commands
% all section headers use the same sans serif family - roboto
%\allsectionsfont{\sffamily\selectfont\roboto\mdseries\bfseries}
\allsectionsfont{\sffamily\selectfont\roboto\mdseries\bfseries}
% Set size and color of section header AL's H1
%\sectionfont{\LARGE\nohang\centering\roboto\textcolor[cmyk]{0.90, 0.54, 0.28, 0.12}}
\sectionfont{\LARGE\nohang\centering\roboto\textcolor[cmyk]{0.90, 0.54, 0.28, 0.12}}
% Set size and color of subsection header AL's H2
%\subsectionfont{\fontsize{18pt}{20pt}\selectfont\roboto\nohang\centering\textcolor[cmyk]{0.90, 0.54, 0.28, 0.12}}
\subsectionfont{\fontsize{18pt}{20pt}\selectfont\roboto\nohang\centering\textcolor[cmyk]{0.90, 0.54, 0.28, 0.12}}
% Set size and color of subsubsection header AL's H3
\subsubsectionfont{\hspace{0pt}\nohang\fontsize{16pt}{18pt}\selectfont\roboto\raggedright\textcolor[cmyk]{0.50, 0.05, 0.0, 0.40}}
% Set size and color of paragraph header AL's H4
%\paragraphfont{\fontsize{14pt}{16pt}\selectfont\robotocondensed\fontseries{bl}\selectfont\textcolor[cmyk]{0.50, 0.05, 0.0, 0.40}}
\paragraphfont{\fontsize{14pt}{16pt}\selectfont\robotocondensed\fontseries{bl}\selectfont\textcolor[cmyk]{0.50, 0.05, 0.0, 0.40}}
\newcolumntype{R}{>{\raggedright\arraybackslash}p{3cm}}

%\definecolor{lightgray}{gray}{0.92}
\fancyhead{}
\pagestyle{fancy}

\parskip = 12pt				% space between paragraphs
\setlength{\extrarowheight}{1.5pt}
\setlength{\arraycolsep}{1.5pt}

%Control over how latex wraps text when justified
\tolerance=1
\emergencystretch=\maxdimen
\hyphenpenalty=10000
\hbadness=10000

%%The following sets up how the pdf displays links and functions in Acrobat.
\hypersetup{
	bookmarks    = true,        % show bookmarks bar?
	unicode      = false,       % non-Latin characters in Acrobat's bookmarks
	pdftoolbar   = true,        % show Acrobat's toolbar?
	pdfmenubar   = true,        % show Acrobat's menu?
	pdffitwindow = false,       % window fit to page when opened
	pdfstartview = {FitH},      % fits the width of the page to the window
	pdfnewwindow = false,        % links in new window
	colorlinks   = false,       % false: boxed links; true: colored links
	linkbordercolor = black,    % the color of the link border color (colorlink =false)
	linkcolor    = blue,        % color of internal links (change box color with linkbordercolor)
	citecolor    = blue,       % color of links to bibliography
	filecolor    = blue,     % color of file links
	urlcolor     = blue,        % color of external links
	pdfborderstyle = {/S/U/W 1} % border style will be underline of width 1pt
}

\newcommand\Tstrut{\rule{0pt}{3.5ex}}       % "top" strut
\newcommand\Bstrut{\rule[-1.7ex]{0pt}{0pt}} % "bottom" strut
\newcommand{\TBstrut}{\Tstrut\Bstrut} % top&bottom struts
\newcolumntype{L}{>{\centering\arraybackslash}m{3in}}

\newcommand{\myparagraph}[1]{\paragraph{#1}\mbox{}\\}

% Create custom font title of document
% note the use of:
%% \textcolor in the title itself
%%  \maketitle BEFORE resetting the \fontfamily
\usepackage{xparse}
\usepackage{xpatch}
\NewDocumentCommand{\TitlePageFont}{}{%
  	\sffamily\bfseries%
  	\fontsize{22pt}{24pt}%
  	\selectfont%
}%
% sets title for pdf reader
%\usepackage[pdftitle={Stock Synthesis User Manual},
%           pdfauthor={Richard D. Methot Jr., Chantel R. Wetzel, Ian G. Taylor, and Kathryn Doering},
%          pdfdisplaydoctitle=true]{hyperref}
\xpretocmd{\maketitle}{\TitlePageFont}{}{}
\title{\textcolor[cmyk]{1.00,0.83,0.41,0.36}{Stock Synthesis User Manual\\ Version 3.30.23.1}}
\author{Richard D. Methot Jr., Chantel R. Wetzel, Ian G. Taylor, Kathryn L. Doering,\\Elizabeth F. Perl, and Kelli F. Johnson\\\\\\NOAA Fisheries\\Seattle, WA}
\date{December 05, 2024}

% ====  Glossary ========================================================
% \usepackage[xindy,acronym]{glossaries}
% \setacronymstyle{long-short}
% \glsdisablehyper
% \makenoidxglossaries 
% \loadglsentries{ss3_glossaries.tex}

\begin{document}
	% ====== Title Page ===================================================
	\maketitle
	% \begin{figure}[ht]
	%     \begin{center}
	%     	\includegraphics[alt={Logo of the National Oceanic and Atmospheric Administration (NOAA)}]{noaalogo.jpg}
	%     \end{center}
	% 	\label{fig:logo}
	% \end{figure}

	\thispagestyle{empty}
	\newpage
	\normalfont % this sets the main font to crimson
	\normalsize %% return the text to 12 point font - otherwise you end up with 22 point font!

		
	% ====== Table of Contents ===================================================
	% \glsaddall	
 	\pagenumbering{roman}
	\tableofcontents
	\thispagestyle{empty}
	\cleardoublepage
	\setcounter{secnumdepth}{0}
	%\setcounter{page}{1}
	\newpage
	\raggedright
	% ====  List of figures ========================================================
	\renewcommand{\headrulewidth}{0pt} % removes the top line across each page
	%\listoffigures
	%\addcontentsline{toc}{section}{\numberline{}List of Figures}
	\cleardoublepage
	% ====== Section 1 - 4 =============================================================
	\pagenumbering{arabic}
	\section{Introduction}
	Fish population (aka ``stock'') assessment models determine the impact of past fishing on the historical and current abundance of the population, evaluate sustainable rates of removals (catch), and project future levels of catch reflecting one or more risk-averse catch rules. These catch rules are codified in regional Fishery Management Plans according to requirements of the Sustainable Fisheries Act. In the U.S., approximately 500 federally managed fish and shellfish populations are managed under approximately 50 Fishery Management Plans. About 200 of these populations are assessed each year, based on a prioritized schedule for their current status. Despite this, many minor species have never been quantitatively assessed. Although the pace is slower than that for weather forecasting, fish stock assessments are operational models for fisheries management.
	\pagebreak
	% ======== Section 5: Converting
	% \input{5converting}
	% % ======== Section 6: Starter File
	% \input{6starter}
	% % ======== Section 7: Forecast File
	% \input{7forecast}
	% % ======== Section 8: Data File
	% \input{8data}
	% % ======== Section 9: Control File
	% \input{9control}
	% % ======== Section 10: Optional Inputs
	% \input{10optional_inputs}
	% % ======== Section 11: Likelihoods
	% \input{11likelihoods}
	% %========= Section 12: Running SS
	% \input{12runningSS3}	
	% % ======== Section 13: Output Files
	% \input{13output}	
	% %========= Section 14: R4SS
	% \input{14r4ss}	
	% %========= Section 15: Special Set-ups
	% \input{15special}
	% %========= Section 16: Essays
	% \input{16essays}
	%========= Glossary
	% \hypertarget{Glossary}{}
	% \addcontentsline{toc}{section}{Glossary}
	% \printnoidxglossary[type=main,title={Glossary},nonumberlist]

	% \hypertarget{Acronyms}{}
	% \addcontentsline{toc}{section}{List of Acronyms}
	% \printnoidxglossary[type=\acronymtype,title={List of Acronyms},nonumberlist]

	%========= Reference Section
	\newpage
	\bibliography{SS3}
	\bibliographystyle{JournalBiblio/cjfas}
	\newpage	
				
\end{document}



